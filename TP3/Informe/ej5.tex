\section{Metaheur\'istica GRASP}
\subsection{Explicaci\'on}
La metaheur\'istica \emph{Greedy Randomized Adaptive Search Procedure} (\textbf{GRASP}), es una mezcla de las dos heur\'isticas previas (vistas en \ref{ej3} y \ref{ej4}). Dicho de manera simple: genera un punto de partida de forma golosa para el algoritmo de b\'usqueda local.\\

La distinci\'on de este algoritmo radica en cómo se construye ``\textit{golosamente}'' la soluci\'on inicial.\\

Como la sigla lo indica, consiste en un algoritmo \textit{Goloso Randomnizado}. Es decir que se escogen candidatos a soluci\'on inicial de forma golosa, pero en vez de elegir al que mejor resultados nos pueda arrojar, elegimos alguno de los que mejor lo har\'a.

Esto da pie a la definici\'on de la RCL (Restricted Candidate List), que es el conjunto de candidatos elegibles para la soluci\'on base.

Se puede formar de varias maneras: elegir un nodo entre los $alpha$-\% mejores en cada etapa hasta completar una soluci\'on v\'alida; elegir uno entre los $alpha$ mejores; o bien uno entre los que cumplan en un $gamma$-\% alguna propiedad.\\

Luego se aplica el algoritmo de b\'usqueda local explicado en el inciso anterior sin modificaciones.\\

Un aspecto tambi\'en diferencial de esta heur\'istica, es que no generamos una \'unica instancia inicial, sino que se toman un n\'umero arbitrario de ellas (seg\'un el criterio de parada) y guardamos la soluci\'on \'optima que encontramos, si en alg\'un momento se mejora, se actualiza.

Las vecindades utilizadas son las mismas que se usan en la heur\'isitca de b\'usqueda local.

El criterio de parada que adoptamos fue el de llegar a una determina cantidad de repeticiones del algoritmo, b\'usqueda de soluci\'on inicial y b\'usqueda local a partir de ella, tal que la respuesta \'optima encontrada, no se viera modificada durante todas ellas. Otra opci\'on podr\'ia haber sido correr un n\'umero fijo de veces y quedarnos con la mejor soluci\'on encontrada; o tambi\'en si conocieramos alguna cota, acercarnos a esta en un determinado porcentaje; o una combinaci\'on de todas.

%Explicar detalladamente el algoritmo implementado. Plantear distintos criterios de parada y de seleccion de la lista de candidatos (RCL) de la heurıstica golosa aleatorizada.
\subsection{Experimentaci\'on}
%Realizar una experimentacion que permita observar los tiempos de ejecucion y la calidad de las soluciones obtenidas. Se debe experimentar variando los valores de los parametros de la metaheurıstica (lista de candidatos, criterios de parada, etc.) y las vecindades utilizadas en la busqueda local. Elegir, si es posible, la configuracion que mejores resultados provea para el grupo de instancias utilizado.
