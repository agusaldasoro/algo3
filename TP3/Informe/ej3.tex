\section{Heur\'istica Constructiva Golosa} \label{ej3}
\subsection{Explicaci\'on}
La heurística constructiva golosa busca, dado un grafo, determinar un conjunto independiente dominante mínimo. 

Para ello, el algoritmo ordena los nodos de acuerdo al grado de cada uno de ellos, de mayor a menor. Este orden se establece al comienzo de la ejecución y no se actualiza.\\

Luego, recorre dichos nodos, y por cada uno de ellos, lo agrega al conjunto solución, y va eliminando a sus vecinos, hasta que no quedan más nodos.\\

De esta manera, el algoritmo siempre obtiene un conjunto independiente (dado que por cada nodo que toma, elimina a sus vecinos) y dominante (dado que por cada nodo que toma, lo agrega al conjunto, y elimina a sus vecinos, es decir, que estos son adyacentes a un nodo del conjunto), pero no puede asegurar que siempre sea mínimo. Eso dependerá del grafo.\\

\textcolor{red}{Falta una explicacion mas detallada (un pseudocodigo, o algo), y aclarar que es goloso porque en cada paso elige la mejor opcion y nosostros definimos MEJOR como el mayor grado total}
%Explicar detalladamente el algoritmo implementado.

\newpage
\subsection{Complejidad Temporal}
\textcolor{red}{Aca no me meto, todavia...}

En lo que respecta a la complejidad temporal, demostraremos a continuación que la misma es $O(n^{2})$.\\

Recordemos que el algoritmo ordena los nodos de mayor grado a menor grado, luego toma el primero de dicha lista, lo agrega al conjunto solución, y lo elimina de la lista junto a sus adyacentes. Luego repite el proceso, tomando el siguiente nodo de la lista de nodos restantes.\\
Dicho esto, definiremos a $f(i,n): \mathbb{N} \rightarrow \mathbb{N}$ como:

$f(i,n)$ = Cantidad de operaciones en el peor caso, teniendo $n$ nodos totales y faltando eliminar $i$.\\
Es decir,\par
\begin{itemize}
	\item$f(0,n) = c$, dado que no falta eliminar ningun nodo, el algoritmo termina, con $c$ alguna cantidad constante de operaciones.\par
	\item $f(i,n) = h*(i-1) + f(i-1,n) + c$, con $h$ la cantidad de vecinos del nodo que estoy viendo, $c$ alguna cantidad constante de operaciones, y $f(i-1,n)$ es el llamado recursivo.\\
\end{itemize}
Expliquemos que significa cada monomio:
\begin{itemize}
	\item $h*(i-1)$: En el peor de los casos, recorre por cada nodo de la lista de adyacencia del nodo tomado, a los demas nodos sin marcar (sin incluir el nodo tomado).
	\item $f(i-1,n)$: En el peor de los casos, no hay nodos de la lista de adyacencia, que no hayan sido eliminados aún.
	\item $c$: Alguna cantidad constante de operaciones.
\end{itemize}
Entonces, querremos demostrar que $(\forall i \in {1..n}) f(i,n) \leq k*i*n + c$ (con $k$ y $c$ alguna constante), pues al momento de ejecutar el algoritmo, se tienen $n$ nodos, y faltan eliminar $n$ nodos, siendo la complejidad del mismo, $O(f(n,n)) = O(n^{2})$; y dado que la cantidad de nodos por borrar no puede ser mayor que la cantidad total de nodos.\\

\newpage
{\large\textbf{Teorema}}\\
Dado $f(i,n): \mathbb{N} \rightarrow \mathbb{N}$ definida como\\
$f(i,n)$ = Cantidad de operaciones  en el peor caso, teniendo $n$ nodos totales y faltando eliminar $i$.\\
$f(0,n) = c$\\
$f(i,n) = h*(i-1) + f(i-1,n) + c$\\
Luego, $(\forall i \in {1..n}) f(i,n) \leq k*i*n + c$.\\

{\large\textbf{Demostración}}\\
Sea $P(i) = f(i,n) \leq k*i*n + c$, demostraremos por inducción global, que $(\forall i \in {1..n}) P(i)$\\
Entonces,\\
\textbf{Caso base:}
\begin{itemize}
    \item[•] $f(1,n) = k \leq c*1*n + k$\\
    Dado que tengo $n$ nodos, y solo falta eliminar uno. $k$ es alguna cantidad constante de operaciones para eliminar el nodo y finalizar el algoritmo.
\end{itemize}
\textbf{Paso inductivo:}\\

$\underbrace{P(1) \wedge P(2) \wedge ... \wedge P(m-1)}_{\text{Hipótesis inductiva}} \Rightarrow \underbrace{P(m)}_{\text{Tesis inductiva}}$\\

Es decir, que por hipótesis inductiva, podemos suponer que vale $f(r,n) \leq c*r*n + k$, con $r \leq m-1$, y queremos ver que $f(m,n) \leq c*m*n + k$.\\
Luego,
\begin{align*}
f(m,n) & = h*(m-1) + f(m-1,n) + k \\
 \stackrel{HI}{\Longrightarrow} f(m,n) & \leq h*(m-1) + (m-1)*n + k \\
 & = h*m - h + m*n - n + k\\
 & = h*(m-1) + n*(m-1) + k\\
 \intertext{Dado que la cantidad de vecinos adyacentes esta acotada por la cantidad de nodos totales, $0 \leq h < n$, luego}
 & \leq n*(m-1)+n*(m-1) + k\\
 & \leq n*m + n*m + k\\
 & = 2*n*m + k\\
 & \leq \underbrace{c*n*m + k}_{\text{Tesis inductiva}} \text{       En particular, con $c = 2$ en este caso}
\end{align*}
\hfill $\blacksquare$

Entonces, $f(i,n) \leq c*i*n + k$, para algún $c$ y $k$ constantes.
Dado que el algoritmo ejecuta, en peor caso, $f(n,n)$ operaciones, su complejidad esta acotada por $f(n,n)$, por lo tanto, tiene complejidad $O(n^{2})$.\\

También es adecuado decir, que el algoritmo tiene complejidad $\Omega(n*log(n))$, dado que debe ordenar los nodos de acuerdo al grado de cada uno, y cualquier algoritmo de ordenamiento basado en el árbol de decisiones no posee mejor complejidad que la mencionada.

%Calcular el orden de complejidad temporal de peor caso del algoritmo.
\subsection{Comparaci\'on de resultados con soluci\'on \'optima}
%Describir instancias de CIDM para las cuales la heuristica no proporciona una solucion optima. Indicar que tan mala puede ser la solucion obtenida respecto de la solucion optima.
\subsection{Experimentaci\'on}
%Realizar una experimentacion que permita observar la performance del algoritmo en terminos de tiempo de ejecucion en funcion de los parametros de la entrada.
