\documentclass[a4paper]{article}
\usepackage[spanish]{babel}
\usepackage[utf8]{inputenc}
\usepackage{charter}   % tipografia
\usepackage{graphicx}
\usepackage{amsmath}
\usepackage{ amssymb }
\usepackage{bm}

%%%%%%%LO AGREGUE%%%%%%%%%%  Y yo lo modifique
%\usepackage{hyperref}
%%%%%%%%%%%%%%%%%%%%%%%%%%

\usepackage[bookmarks = true, colorlinks=true, linkcolor = black, citecolor = black, menucolor = black, urlcolor = blue]{hyperref} 




%\usepackage{makeidx}
\usepackage{paralist} %itemize inline
\usepackage[lined,boxed,commentsnumbered]{algorithm2e}
%\usepackage[ruled,vlined]{algorithm2e}
%\usepackage{float}
%\usepackage{amsmath, amsthm, amssymb}
%\usepackage{amsfonts}
%\usepackage{sectsty}
%\usepackage{charter}
%\usepackage{wrapfig}
%\usepackage{listings}
%\lstset{language=C}
\usepackage[usenames,dvipsnames]{xcolor}



\input{codesnippet}
\input{page.layout}
% \setcounter{secnumdepth}{2}
\usepackage{underscore}
\usepackage{caratula}
%\usepackage{url}
\usepackage{hyperref}

% ******************************************************** %
%              TEMPLATE DE INFORME ORGA2 v0.1              %
% ******************************************************** %
% ******************************************************** %
%                                                          %
% ALGUNOS PAQUETES REQUERIDOS (EN UBUNTU):                 %
% ========================================
%                                                          %
% texlive-latex-base                                       %
% texlive-latex-recommended                                %
% texlive-fonts-recommended                                %
% texlive-latex-extra?                                     %
% texlive-lang-spanish (en ubuntu 13.10)                   %
% texlive-science										  %
% ******************************************************** %


\begin{document}


\thispagestyle{empty}
\materia{Algoritmos y Estructuras de Datos III}
\submateria{Primer Cuatrimestre de 2015}
\titulo{Trabajo Práctico III}
%\subtitulo{subtitulo del trabajo}
\integrante{Aldasoro Agustina}{86/13}{agusaldasoro@gmail.com}
\integrante{Noriega Francisco}{660/12}{frannoriega.92@gmail.com}
\integrante{Zimenspitz Ezequiel}{155/13}{ezeqzim@gmail.com}
\integrante{Zuker Brian}{441/13}{brianzuker@gmail.com}


\maketitle
\newpage

\thispagestyle{empty}
\vfill
\begin{abstract}
El objetivo de este trabajo es plantear soluciones a la problem\'atica de \emph{Conjunto Independiente Dominante M\'inimo}, para lo cual se desarrolla un algoritmo exacto que calcula la soluci\'on \'optima y tambi\'en heur\'isticas con el fin de abordar la misma problem\'atica.

El c\'odigo de este trabajo pr\'actico presenta una funci\'on \texttt{main} que permite correr cualquiera de los algoritmos desarrollados bajo el mismo input e incluso hacerlo veces consecutivas. Cada uno recibe par\'ametros necesarios para reutilizar el cod\'igo. B\'usqueda Local utiliza Greedy para generar instancias iniciales, GRASP aprovecha a la b\'usqueda local y una adaptaci\'on del greedy. Ver secci\'on \ref{anexo}.\\

Para compilar se usa \texttt{g++ -o main correrCIDM.cpp -std=c++11}

Esta flag se a\~nade con el fin de poder utilizar funciones de medici\'on para los tiempos de ejecuci\'on dentro de la experimentaci\'on.


\end{abstract}

\thispagestyle{empty}
\vspace{3cm}
\tableofcontents


\newpage


%\normalsize
\newpage
\section{Introducci\'on al problema}
\subsection{Conjunto Independiente Dominante M\'inimo (CIDM)}

Sea $G = (V, E)$ un grafo simple. Un conjunto $D \subseteq V$ es un \emph{conjunto dominante} de $G$ si todo v\'ertice de $G$ est\'a en $D$ o bien tiene al menos un vecino que est\'a en $D$. 

Por otro lado, un conjunto $I \subseteq V$ es un \emph{conjunto independiente} de $G$ si no existe ning\'un eje de $E$ entre dos v\'ertices de $I$. 

Definimos entonces un \emph{conjunto independiente dominante} de $G$ como un conjunto independiente que a su vez es un conjunto dominante del grafo $G$.\\

El problema de Conjunto Independiente Dominante M\'inimo (CIDM) consiste en hallar un conjunto independiente dominante de $G$ con m\'inima cardinalidad.

\subsection{Paralelismo con ``El se\~nor de los caballos''}

\textcolor{red}{Aca va lo de Noriega y sus super scripts (?)}

\subsection{Todo conjunto independiente maximal es un conjunto dominante}

 Sea $G = (V, E)$ un grafo simple, un \emph{conjunto independiente} de $I \subseteq V$ se dice \emph{maximal} si no existe otro conjunto independiente $J \subseteq V$ tal que $I \subset J$, es decir que $I$ est\'a incluido estrictamente en $J$. \\
 
 Todo conjunto independiente maximal es un \emph{conjunto dominante}.
 
\subsubsection*{Demostraci\'on}

Sean $G = (V, E)$ grafo simple, $I \subseteq V$ \emph{conjunto independiente maximal}.\\

Quiero ver que $I$ es un \emph{conjunto dominante}: 

Lo que es equivalente a:
Para todo v\'ertice $v \in V$, o bien v pertenece a I, o bien alg\'un vecino de v pertenece a I.
%($\forall$ v $\in$ V) (v $\in$ I $\vee$ $\exists$ w $\in$ adyacentes(v) tal que w $\in$ I)\\
Como I es independiente, no existe ning\'un eje de E entre v\'ertices de I; por ser I maximal, no existe ning\'un v\'ertice w de V que no pertenezca a I, tal que w no sea adyacente a alguno de I.\\
Entonces:
\begin{itemize}
\item si E = $\emptyset$, I = V y por lo tanto cada elemento de v pertenece a I trivialmente
\item si E $\neq$ $\emptyset$, todos los v\'ertices v $in$ V que no pertenezcan a I, no lo hacen porque ya existe w $in$ I que verifica que el eje (v,w) $in$ E. Luego, cada u v\'ertice de V o bien pertenece a I, o bien es adyacente a alg\'un elemento de I. Por lo tanto I es un conjunto dominante.
\end{itemize}
\textcolor{blue}{se fijan que onda?}
\textcolor{red}{Demostracion magica negra turbia}

\subsection{Situaciones de la vida real}

Situaciones de la vida real que puedan modelarse utilizando CIDM:

\textcolor{red}{Otra vez Noriega salvando las papas del fuego..} lo del aula en el examen y eso
\newpage
\section{Zombieland II}
\subsection{Descripci\'on de la problem\'atica}
\subsection{Resoluci\'on propuesta y justificaci\'on}
\subsection{An\'alisis de la complejidad}
\subsection{C\'odigo fuente}
\subsection{Experimentaci\'on}
\subsubsection{Constrastaci\'on Emp\'irica de la complejidad}
Para realizar los experimentos, se consideró como peor caso, aquel en el que se debiera recorrer cada nodo de la ciudad, tantas veces como cantidad de soldados se tuviese a disposicion, siendo ese caso, en el que se recorrerían $s.n.m$ elementos, coiniciendo así con la complejidad teórica $O(s.n.m)$.\\
Aunque ese el caso deseable, generar dicha instancia para los distintos valores de $s$, $n$ y $m$, resultó tan dificultoso como resolver el problema en cuestión.\\
Por ello, se decidio generar instancias con una cantidad aleatoria de zombies por cuadra. Ésto permitió generar instancias automáticamente, para casos de prueba grandes, pero trajo consigo los siguientes problemas:
\begin{enumerate}
	\item No se puede determinar si existe un camino desde el punto de inicio hasta el búnker.
	\item No se puede determinar, si al existir un camino, este será único.
	\item No se puede determinar cuantos soldados van a llegar al búnker.
	\item No se puede determinar cuantos caminos adulteran la cantidad de soldados, ni cuantos soldados mueren en cada uno de ellos.
\end{enumerate}

El resultado del generador de instancias, proporcionaba un laberinto como el siguiente:

\begin{center}
\includegraphics[width=7cm,keepaspectratio=yes]{imagenes/ej2/maze.png}
\end{center}

Los pasajes del laberinto son los caminos donde hay entre $0$ y $s+1$ zombies, y las paredes, caminos donde hay entre $0$ y $s$.2.\\
Estos valores aleatorios fueron tomados adrede, por un lado, para que incluso el mejor camino (alguno de los pasajes que conducen a la salida), tuviera perdida de soldados, pero intentando que éstas sean mínimas, para aumentar las probabilidades de que puedan efectivamente llegar al búnker.
Las paredes, no son un obstaculo, dado que podrían terminar siendo un pasaje, pero la probabilidad de que eso ocurra es muy baja.\\
Aún asi, el $s$ tomado en el algoritmo aleatorio es el que se recibe en la entrada. Esto significa, que se configura al principio, y despues no se actualiza con la cantidad de soldados despues de una batalla.\\
Por ende, no podemos determinar el porcentaje de soldados totales que llegarán, pero sí podemos saber el porcentaje que sobrevivirá a la primer batalla.\\
Así, la cantidad de soldados que sobreviviran a la primer batalla es:
\begin{itemize}
	\item 95\% en caso de que el camino sea un pasaje.
	\item 50\% en casos de que el camino sea una pared.
\end{itemize}
Estos rangos aleatorios también se obtuvieron experimentando, hasta lograr valores que ocasionaran bajas pero que no impidieran a los soldados llegar al búnker en la mayoría de los casos. Dichos experimentos exceden el interés de esta sección y no serán discutidos.\\

Pese a dificultades expuestas, se decidió proceder con dichas instancias, ya que lograban generar una fluctuación en la cantidad de soldados, y esto, si bien lejos del peor caso propuesto, resultó ser una aproximación que nos resultó suficiente.\\

Por lo dicho anteriormente, se realizaron experimentos sobre los siguientes dos casos:
\begin{itemize}
	\item \textbf{Cero zombies}: En esta instancia, la cantidad de zombies en cada cuadra es cero, con lo cual, el algoritmo recorre toda la ciudad, y se queda con cualquier camino.
	\item \textbf{Zombies aleatorios}: Es la instancia explicada anteriormente.
\end{itemize}

También es necesario aclarar, que los experimentos se realizaron sobre ciudades cuadradas, con puntos de inicio y llegada en los extremos opuestos de la ciudad, y con cantidad de soldados iniciales igual a 20.\\

A continuación se detallan los experimentos y sus resultados.
Debido al tamaño de las instancias de prueba, los inputs de dichos experimentos no fueron adjuntados.
\begin{center}
	\begin{tabular}{|c|c|c|}
	\hline
	Experimento & \textbf{Cero zombies} & \textbf{Zombies aleatorios}\\
	\hline
	\hline
	Tamaño de la ciudad & \multicolumn{2}{|c|}{Soldados al final}\\
	\hline
	50x50 & 20 & 6\\
	\hline
	100x100 & 20 & 19\\
	\hline
	250x250 & 20 & 17\\
	\hline
	500x500 & 20 & 17\\
	\hline
	750x750 & 20 & 16\\
	\hline
	1000x1000 & 20 & 12\\
	\hline
	1250x1250 & 20 & 20\\
	\hline
	1500x1500 & 20 & 7\\
	\hline
	1750x1750 & 20 & 14\\
	\hline
	2000x2000 & 20 & 20\\
	\hline
	\end{tabular}
\end{center}

Dado que los tiempos de ejecución para ambos experimentos varían ampliamente, primero analizaremos el experimento de \textbf{Cero zombies}.

\includegraphics[width=15cm,keepaspectratio=yes]{imagenes/ej2/czneto.png}

Como se puede apreciar, al no haber zombies, no existe ningun camino en el cual mueran soldados, por lo que los tiempos se incrementan acorde a la dimension de la ciudad, y al ser cuadradas, crece polinomialmente.

\includegraphics[width=15cm,keepaspectratio=yes]{imagenes/ej2/czyza.png}

Aquí se pueden apreciar las diferencias de tiempos. Ésta radica en el hecho de que \textbf{Zombies aleatorios} posee caminos en los cuales los soldados mueren. Por ello, el algoritmo deberá recorrer, en peor caso, $s$ veces la ciudad entera.\\

Se procedió, entonces, a dividir los resultados de \textbf{Zombies aleatorios} por la cantidad de soldados iniciales con los que se ejecutó el algoritmo a fin de que quede reflejado, que en ese caso, los tiempos serán muy similares a los de \textbf{Cero zombies}. Esto se debe a que ya no son los tiempos de recorrer $s$ veces la ciudad, sino de recorrerla una sola vez.
Sin embargo, es necesario aclarar que la similitud que se intenta mostrar, es aproximada, ya que no podemos asegurar que efectivamente, el algoritmo recorre $s$ veces la ciudad. Tal es el peor caso, y como hemos expuesto en un principio, no podemos asegurar que sea el que realmente ocurre.

\includegraphics[width=15cm,keepaspectratio=yes]{imagenes/ej2/zaczados.png}

Manteniendo los resultados de \textbf{Zombies aleatorios} divididos por $s$, finalmente dividimos los resultados tanto de \textbf{Cero zombies} como de \textbf{Zombies aleatorios}, por $n$, y en una instancia aparte, por $n.m$.
De esta manera, dado que en ambos tienen $s$ igual a 1, solo queda ver que al dividir por $n$, los resultados se aproximan a una función lineal, y que al dividirlos por $n.m$, se aproximan a una constante.\\

Como solo nos interesa la relación, y no la función exacta ni la constante, multiplicamos los resultados de dichas divisiones por 100, para que los valores sean más claros y visibles.

\includegraphics[width=15cm,keepaspectratio=yes]{imagenes/ej2/linealizacion.png}

\includegraphics[width=15cm,keepaspectratio=yes]{imagenes/ej2/constantizacion.png}

Se puede ver que la experimentación se corresponde con la teoría, y que la complejidad, en peor caso, es $O(s.n.m)$.


\newpage
\section{Heur\'istica Constructiva Golosa}
\subsection{Explicaci\'on}
%Explicar detalladamente el algoritmo implementado.
\subsection{Complejidad Temporal}
%Calcular el orden de complejidad temporal de peor caso del algoritmo.
\subsection{Comparaci\'on de resultados con soluci\'on \'optima}
%Describir instancias de CIDM para las cuales la heuristica no proporciona una solucion optima. Indicar que tan mala puede ser la solucion obtenida respecto de la solucion optima.
\subsection{Experimentaci\'on}
%Realizar una experimentacion que permita observar la performance del algoritmo en terminos de tiempo de ejecucion en funcion de los parametros de la entrada.

\newpage
\section{Heur\'istica de B\'usqueda Local} \label{ej4}

Un algoritmo de Búsqueda Local consiste en dos simples pasos: elegir una solución inicial y luego, \textcolor{red}{Aca explicarlo bien, esta horrible} iterar


modificarla (``mejorandola''), reemplazándola paso a paso con distintas soluciones que pertenezcan a la vecindad de la misma.


Para cada solución factible s $\epsilon$ S se define N(s) como el conjunto de
``soluciones vecinas'' de s. Un procedimiento de busqueda local toma una solución inicial s e iterativamente la mejora reemplazándola por otra solución mejor del conjunto N(s), hasta llegar a un óptimo local.

 Sea s $\epsilon$ S una solución inicial
 
 Mientras exista s $\epsilon$ N(s) con f (s) $>$ f (s)
 
 s $\leftarrow$ s


\subsection{Explicaci\'on}
%Explicar detalladamente el algoritmo implementado. Plantear al menos dos vecindades distintas para la busqueda y al menos dos soluciones iniciales.

Considerando el problema a tratar, establecimos nuestros criterios para encontrar las soluciones iniciales y las vecindades.\\


\subsubsection{Elección de Solución Inicial}

Al momento de seleccionar la solución Inicial, determinamos dos criterios.

\subsubsection*{Criterio I Solución Inicial: Golosa}

Se realiza una ejecución del algoritmo Goloso de la Sección \ref{ej3}.\\

Esto quiere decir, se ordenan los nodos por grado de manera decreciente. Se eligen los nodos de a uno (de mayor a menor), de modo que al elegir un nodo se descartan sus vecinos para sus futuras elecciones.

\subsubsection*{Criterio II Solución Inicial: Secuencial}

Los nodos al ser ingresados como parámetro del algoritmo tienen como identificador un número entre $0$ y $n-1$. El orden que vamos a utilizar para recorrerlos es el que haya sido dado cuando fueron ingresados como parámetro.\\

Lo primero que realizamos es tomar al nodo $0$ y considerarlo parte de la solución. Se descartan todos los nodos vecinos a él y se continúa el proceso con el nodo que tenga menor número de \textit{id}.

De este modo se forma un conjunto solución tal que en cada paso añade al nodo disponible que tenga su identificador número menor.

\subsubsection{Elección de Solución Inicial}

Al momento de seleccionar la solución Inicial, determinamos dos criterios.

\subsubsection*{Criterio I Solución Inicial}

\subsubsection*{Criterio I Solución Inicial}





\textcolor{red}{poner cómo ejecutar cada uno ed los criterios}

\subsection{Complejidad Temporal}
%Calcular el orden de complejidad temporal de peor caso de una iteracion del algoritmo de busqueda local (para las vecindades planteadas). Si es posible, dar una cota superior para la cantidad de iteraciones de la heurıstica.
\subsection{Experimentaci\'on}
%Realizar una experimentacion que permita observar la performance del algoritmo comparando los tiempos de ejecucion y la calidad de las soluciones obtenidas, en funcion de las vecindades y las soluciones iniciales utilizadas y elegir, si es posible, la configuracion que mejores resultados provea para el grupo de instancias utilizado.
\newpage
\section{Metaheur\'istica GRASP}
\subsection{Explicaci\'on}
%Explicar detalladamente el algoritmo implementado. Plantear distintos criterios de parada y de seleccion de la lista de candidatos (RCL) de la heurıstica golosa aleatorizada.
\subsection{Experimentaci\'on}
%Realizar una experimentacion que permita observar los tiempos de ejecucion y la calidad de las soluciones obtenidas. Se debe experimentar variando los valores de los parametros de la metaheurıstica (lista de candidatos, criterios de parada, etc.) y las vecindades utilizadas en la busqueda local. Elegir, si es posible, la configuracion que mejores resultados provea para el grupo de instancias utilizado.

\newpage
\section{Comparaci\'on entre todos los m\'etodos} \label{ej6}

\subsection{Comparaci\'on}
Aqu\'i compararemos todos los algoritmos contra el exacto, que es el que nos da la soluci\'on \'optima en todos los casos y veremos cu\'al es la diferencia existente.\\

Comenzando con 45 ejes fijos y variando los nodos, los resultados fueron:\\

\begin{tabular}{| l | l | l | l | l |}
 \hline
Nodos&Exacto&Local&Grasp&Greedy \\ \hline
10&1&1&1&1 \\ \hline
12&2&2&2&2 \\ \hline
14&2&2&2&2 \\ \hline
16&3&3&3&3 \\ \hline
18&4&4&4&4 \\ \hline
20&5&7&7&7 \\ \hline
22&5&5&5&5 \\ \hline
24&7&7&7&7 \\ \hline
26&7&8&8&9 \\ \hline
28&9&10&10&11 \\ \hline
30&12&14&14&14 \\ \hline
\end{tabular}

Reci\'en pueden empezar a notarse las diferencias cuando la cantidad de nodos es mayor que 26.\\

Continuamos con 90 ejes fijos:

\begin{tabular}{| l | l | l | l | l |}
 \hline
Nodos&Exacto&Local&Grasp&Greedy \\ \hline
15&1&1&1&2 \\ \hline
17&2&2&2&2 \\ \hline
19&3&3&3&3 \\ \hline
21&3&3&4&4 \\ \hline
23&4&4&4&5 \\ \hline
25&3&3&4&6 \\ \hline
27&5&5&5&8 \\ \hline
29&6&7&8&7 \\ \hline
31&6&9&8&9 \\ \hline
33&7&8&8&9 \\ \hline
\end{tabular}

Nuevamente, los algoritmos suelen coincidir con el exacto cuando la cantidad de nodos es baja, pero luego comienzan a verse disparidades.\\

Ahora fijaremos los nodos y variaremos la cantidad de ejes.\\

Con 10 nodos fijos:\\
\begin{tabular}{| l | l | l | l | l |}
 \hline
Ejes&Exacto&Local&Grasp&Greedy \\ \hline
0&10&10&10&10 \\ \hline
5&6&6&6&6 \\ \hline
10&4&4&4&5 \\ \hline
15&3&3&3&4 \\ \hline
20&3&3&3&3 \\ \hline
25&2&2&2&3 \\ \hline
30&2&2&2&3 \\ \hline
35&2&2&2&2 \\ \hline
40&1&1&1&1 \\ \hline
45&1&1&1&1 \\ \hline
\end{tabular}

Como la cantidad de nodos es baja, era bastante probable que los algoritmos consigan la respuesta \'optima. De todas formas, podemos ver que el Greedy falla en algunos casos.\\

Con 20 nodos fijos:\\
\begin{tabular}{| l | l | l | l | l |}
 \hline
Ejes&Exacto&Local&Grasp&Greedy \\ \hline
0&20&20&20&20 \\ \hline
5&15&15&15&15 \\ \hline
10&11&11&11&11 \\ \hline
15&10&12&10&12 \\ \hline
20&8&8&8&8 \\ \hline
25&8&8&8&8 \\ \hline
30&6&7&7&8 \\ \hline
35&6&6&6&7 \\ \hline
40&5&6&6&6 \\ \hline
45&5&6&6&6 \\ \hline
50&4&4&4&5 \\ \hline
55&5&5&5&5 \\ \hline
60&4&4&4&4 \\ \hline
65&5&5&5&5 \\ \hline
70&3&3&3&3 \\ \hline
75&3&3&3&4 \\ \hline
\end{tabular}\\

Con 30 nodos fijos:\\
\begin{tabular}{| l | l | l | l | l |}
 \hline
Ejes&Exacto&Local&Grasp&Greedy \\ \hline
0&30&30&30&30 \\ \hline
5&25&25&25&25 \\ \hline
10&21&21&22&21 \\ \hline
15&17&19&20&19 \\ \hline
20&16&16&19&16 \\ \hline
25&16&15&15&15 \\ \hline
30&13&12&12&12 \\ \hline
35&10&12&13&13 \\ \hline
40&10&12&12&12 \\ \hline
45&10&10&10&11 \\ \hline
50&11&11&13&12 \\ \hline
55&9&9&11&10 \\ \hline
60&8&11&11&11 \\ \hline
65&9&10&10&10 \\ \hline
70&8&9&9&10 \\ \hline
75&7&9&9&10 \\ \hline
80&6&6&6&6 \\ \hline
85&6&7&7&8 \\ \hline
90&6&7&7&7 \\ \hline
95&5&7&7&9 \\ \hline
100&5&7&7&7 \\ \hline
105&5&6&6&6 \\ \hline
110&5&6&6&8 \\ \hline
115&5&5&5&6 \\ \hline
120&5&5&5&5 \\ \hline
\end{tabular}





\subsection{Resultados}
\subsubsection{Nodos fijos}
Aqu\'i se comparan los resultados de los 3 algoritmos (Grasp, Greedy y Local), manteniendo los nodos fijos. Nuestros tests fueron con nodos fijos desde 100 hasta 1000 y para cada una de esas instancias,
variando los ejes de a 100, comenzando con 1000 ejes.\\

En estos gr\'aficos, vamos a buscar, para cada cambio en los ejes, cu\'al fue el que menos nodos utiliza en su respuesta. \'Este ser\'a el mejor en cuanto a resultados.\\

Mostraremos los gr\'aficos con 100, 300, 600 y 900 nodos fijos:

\begin{center}
 	\includegraphics[width=13cm, keepaspectratio=yes]{imagenes/6/100NodosFijos.png}

 	\includegraphics[width=13cm, keepaspectratio=yes]{imagenes/6/300NodosFijos.png}
 
 	\includegraphics[width=13cm, keepaspectratio=yes]{imagenes/6/600NodosFijos.png}
 
 	\includegraphics[width=13cm, keepaspectratio=yes]{imagenes/6/900NodosFijos.png}
\end{center}
 
Analizando estos gr\'aficos, podemos ver r\'apidamente que el algoritmo Greedy nos devuelve en todos los casos una soluci\'on peor que los otros dos, ya que utiliza mayor cantidad de nodos.\\
Por otra parte, se puede ver cierta ventaja del Grasp para los casos m\'as chicos (100, 300 nodos), pero para los casos grandes (600, 900 nodos), se ve que el Local es mejor en casi todos los casos.\\

No podemos distinguir claramente cu\'al es la mejor opci\'on entre estos dos, ya que los valores son bastante cambiantes a lo largo de los 4 gr\'aficos, por lo que buscamos el promedio de nodos que utiliza
cada algoritmo. Los resultados fueron los siguientes:\\

\begin{tabular}{| l | l | l | l |}
   \hline
   Nodos & Grasp & Local & Greedy\\ \hline
   100 & 6 & 6.55 & 7.65 \\ \hline
   300 & 46.59 & 47.25 & 53.65 \\ \hline
   600 & 149.255 & 145.65 & 164.3 \\ \hline
   900 & 289.925 & 277.95 & 307.55 \\
   \hline
\end{tabular}

Estos promedios confirman los datos que se pod\'ian apreciar en los gr\'aficos, es decir, que el Greedy utiliza siempre mayor cantidad de nodos que los otros, que Grasp es mejor en los casos chicos y
que a medida que crece la cantidad de nodos, es preferible el Local.\\

También se analizaron los tiempos de ejecución en las mismas instancias, cuyos resultados fueron los siguientes:

\begin{center}
 	\includegraphics[width=13cm, keepaspectratio=yes]{imagenes/coliseo/Fixnode/100.png}

 	\includegraphics[width=13cm, keepaspectratio=yes]{imagenes/coliseo/Fixnode/300.png}

 	\includegraphics[width=13cm, keepaspectratio=yes]{imagenes/coliseo/Fixnode/600.png}

 	\includegraphics[width=13cm, keepaspectratio=yes]{imagenes/coliseo/Fixnode/900.png}
\end{center}

Se puede observar, que los algoritmos empeoran a mayor cantidad de nodos, pero debe notarse, que en general, Local y Greedy permanecen más bien constantes, mientras que Grasp empeora a mayor cantidad de ejes tiene, en grafos más grandes. 
Por lo tanto, y conforme al análisis anterior, queda aún más claro que para casos grandes, Local es la mejor alternativa al momento de fijar nodos y variar ejes, sea por tiempos de ejecución, como por mejor resultado.

\subsubsection{Ejes Fijos}
Para continuar, realizamos los mismos tests, pero manteniendo los ejes fijos y variando la cantidad de nodos. Los tests realizados fueron con ejes fijos desde 0 hasta 1000 y para cada instancia,
los nodos var\'ian de 50 a 1000.\\

Mostraremos los resultados obtenidos con 300, 600 y 900 ejes fijos:\\

   \begin{center}
 	\includegraphics[width=13cm, keepaspectratio=yes]{imagenes/6/300EjesFijos.png}

 	\includegraphics[width=13cm, keepaspectratio=yes]{imagenes/6/600EjesFijos.png}

 	\includegraphics[width=13cm, keepaspectratio=yes]{imagenes/6/900EjesFijos.png}
   \end{center}
 
A primera vista, se puede ver que para casi todas las instancias el algoritmo de B\'usqueda Local es el que menos nodos utiliza en su soluci\'on. 
En cuanto al Greedy y el Grasp, las diferencias son chicas y alternadas con 300 ejes, pero con 600 y 900 ejes, Greedy aparenta ser el peor.\\

Para corroborar estos datos, tomamos nuevamente el promedio de todas las soluciones:\\

\begin{tabular}{| l | l | l | l |}
   \hline
   Nodos & Grasp & Local & Greedy\\ \hline
   300 & 337.61 & 332 & 338.15 \\ \hline
   600 & 256.165 & 250.4 & 261.8 \\ \hline
   900 & 207.44 & 203.1 & 217.9 \\
   \hline
\end{tabular}

Como preve\'iamos, estos promedios confirman nuestro an\'alisis.

Nuevamente, para fortalecer nuestra conclusión, se analizaron los tiempos de ejecución en las mismas instancias, cuyos resultados fueron los siguientes:

\begin{center}
 	\includegraphics[width=13cm, keepaspectratio=yes]{imagenes/coliseo/Fixedge/0.png}

 	\includegraphics[width=13cm, keepaspectratio=yes]{imagenes/coliseo/Fixedge/300.png}

 	\includegraphics[width=13cm, keepaspectratio=yes]{imagenes/coliseo/Fixedge/600.png}

 	\includegraphics[width=13cm, keepaspectratio=yes]{imagenes/coliseo/Fixedge/900.png}
\end{center}

Como era esperado, Grasp insume mayor tiempo que Local y Greedy para cualquier caso.
Considerando los resultados y los tiempos de ejecución, concluímos que Local es la mejor alternativa, si se mantienen ejes y se varían nodos.

\subsubsection{Grafo bipartito completo}

Dado que los resultados siempre son \'optimos, se omitieron los gr\'aficos de resultados para este caso.

Analizando los tiempos de ejecución sobre instancias de 100, 300, 600 y 900 nodos, se obtuvieron los siguientes resultados:

\begin{center}
 	\includegraphics[width=13cm, keepaspectratio=yes]{imagenes/coliseo/Bipartite1.png}

 	\includegraphics[width=13cm, keepaspectratio=yes]{imagenes/coliseo/Bipartite2.png}
\end{center}

Para el segundo gráfico, se omitió la ejecución del Grasp, dado que los tiempos de ejecución del mismo para los grafos bipartitos completos de esa magnitud, eran demasiado altos, 
lo cual era predecible pues debe ejecutar la b\'usqueda local 10 veces, es decir que su ejecuci\'on consume \textit{10x(tiempo de b\'usqueda local)}.


Dado que los tres algoritmos siempre obtienen la solución óptima para la familia de grafos bipartitos completos, concluímos que el Greedy es el mejor de los tres para estos casos, ya que es el que menor tiempo insume. 

\subsubsection{Ciclo simple}

Por \'ultimo, analizaremos qu\'e ocurre cuando el grafo de entrada es un ciclo. Para ello fuimos generando distintos ciclos, comenzando con 100 nodos y aumentando de a 100 hasta 1000 nodos.\\
Los resultados obtenidos fueron:\\

\begin{center}
 \includegraphics[width=13cm, keepaspectratio=yes]{imagenes/6/Ciclos.png}
\end{center}

Se aprecia r\'apidamente que el Greedy es el peor de los 3 y que los resultados entre el Grasp y Local son muy parejos, siendo Local un poco mejor en los casos m\'as grandes.\\

A su vez, los tiempos de ejecucion sobre las mismas instancias, se obtuvieron los siguientes resultados:

\begin{center}
 	\includegraphics[width=13cm, keepaspectratio=yes]{imagenes/coliseo/Cicle.png}
\end{center}

Analizando tanto tiempos como resultados obtenidos, concluimos que el algoritmo Local es el m\'as conveniente cuando el grafo es un ciclo, ya que obtiene resultados mucho mejores al Greedy a pesar
de ser un poco m\'as lento. Adem\'as, es mejor que Grasp tanto en tiempo de ejecuci\'on como en el resultado.

\newpage

\section{Anexo}\label{anexo}
Al momento de ejecutar el \texttt{main} se le deben pasar los siguientes par\'ametros acorde a lo deseado:

\begin{itemize}
\item \textbf{0}: \emph{Algortimo Exacto};
\item \textbf{1}: \emph{Heur\'istica Greedy} (con par\'ametros alpha = 0, conAlpha = true);
\item \textbf{2}: \emph{Heur\'istica B\'usqueda local} (soluci\'on inicial por orden de nomenclatura (greedy = false), vecindad 2x1 (vecindad = true));
\item \textbf{3}: \emph{Heur\'istica B\'usqueda local} (soluci\'on inicial greedy (greedy = true), vecindad 2x1 (vecindad = true), alpha = 0);
\item \textbf{4}: \emph{Heur\'istica B\'usqueda local} (soluci\'on inicial por orden de nomenclatura (greedy = false), vecindad 3x1 (vecindad = false));
\item \textbf{5}: \emph{Heur\'istica B\'usqueda local} (soluci\'on inicial greedy (greedy = true), vecindad 3x1 (vecindad = false), alpha = 0);
\item \textbf{6}: \emph{Heur\'istica GRASP} (soluci\'on inicial por pocentaje de mejores (conAlpha = true), vecindad 2x1 (vecindad = true), alpha = input);
\item \textbf{7}:\emph{ Heur\'istica GRASP} (soluci\'on inicial por pocentaje de mejores (conAlpha = true), vecindad 3x1 (vecindad = false), alpha = input);
\item \textbf{8}:\emph{ Heur\'istica GRASP }(soluci\'on inicial por cantidad de mejores (conAlpha = false), vecindad 2x1 (vecindad = true), alpha = input);
\item \textbf{9}: \emph{Heur\'istica GRASP} (soluci\'on inicial por cantidad de mejores (conAlpha = false), vecindad 3x1 (vecindad = false), alpha = input);
\item \textbf{i}: \emph{Imprime} la lista de adyacencia del grafo pasado por input;
\item \textbf{q}: \emph{Finaliza} la ejecuci\'on;
\end{itemize}


% \section{Objetivos generales}

% El objetivo de este Trabajo Práctico es ...


% \section{Contexto}

% \begin{figure}
%   \begin{center}
% 	\includegraphics[scale=0.66]{imagenes/logouba.jpg}
% 	\caption{Descripcion de la figura}
% 	\label{nombreparareferenciar}
%   \end{center}
% \end{figure}


% \paragraph{\textbf{Titulo del parrafo} } Bla bla bla bla.
% Esto se muestra en la figura~\ref{nombreparareferenciar}.




%Habra que insertar el enunciado???
% %\section{Enunciado y solucion} 
% %\input{enunciado}

% \section{Conclusiones y trabajo futuro}

\end{document}

