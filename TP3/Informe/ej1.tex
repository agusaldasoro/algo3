\section{Introducci\'on al problema}
\subsection{Conjunto Independiente Dominante M\'inimo (CIDM)}

Sea $G = (V, E)$ un grafo simple. Un conjunto $D \subseteq V$ es un \emph{conjunto dominante} de $G$ si todo v\'ertice de $G$ est\'a en $D$ o bien tiene al menos un vecino que est\'a en $D$. 

Por otro lado, un conjunto $I \subseteq V$ es un \emph{conjunto independiente} de $G$ si no existe ning\'un eje de $E$ entre dos v\'ertices de $I$. 

Definimos entonces un \emph{conjunto independiente dominante} de $G$ como un conjunto independiente que a su vez es un conjunto dominante del grafo $G$.\\

El problema de Conjunto Independiente Dominante M\'inimo (CIDM) consiste en hallar un conjunto independiente dominante de $G$ con m\'inima cardinalidad.

\subsection{Paralelismo con ``El se\~nor de los caballos''}

\textcolor{red}{Aca va lo de Noriega y sus super scripts (?)}

\subsection{Todo conjunto independiente maximal es un conjunto dominante}

 Un \emph{conjunto independiente} de $I \subseteq V$ se dice \emph{maximal} si no existe otro conjunto independiente $J \subseteq V$ tal que $I \subset J$, es decir que $I$ est\'a incluido estrictamente en $J$. Todo conjunto independiente maximal es un \emph{conjunto dominante}.
 
\subsubsection*{Demostraci\'on}

\textcolor{red}{Demostracion magica negra turbia}

\subsection{Situaciones de la vida real}

Situaciones de la vida real que puedan modelarse utilizando CIDM:

\textcolor{red}{Otra vez Noriega salvando las papas del fuego..} lo del aula en el examen y eso