\section{Introducci\'on al problema}
\subsection{Conjunto Independiente Dominante M\'inimo (CIDM)}

Sea $G = (V, E)$ un grafo simple. Un conjunto $D \subseteq V$ es un \emph{conjunto dominante} de $G$ si todo v\'ertice de $G$ est\'a en $D$ o bien tiene al menos un vecino que est\'a en $D$. 

Por otro lado, un conjunto $I \subseteq V$ es un \emph{conjunto independiente} de $G$ si no existe ning\'un eje de $E$ entre dos v\'ertices de $I$. 

Definimos entonces un \emph{conjunto independiente dominante} de $G$ como un conjunto independiente que a su vez es un conjunto dominante del grafo $G$.\\

El problema de Conjunto Independiente Dominante M\'inimo (CIDM) consiste en hallar un conjunto independiente dominante de $G$ con m\'inima cardinalidad.

\subsection{Paralelismo con ``El se\~nor de los caballos''}

\textcolor{red}{Aca va lo de Noriega y sus super scripts (?)}

\subsection{Todo conjunto independiente maximal es un conjunto dominante}

 Un \emph{conjunto independiente} de $I \subseteq V$ se dice \emph{maximal} si no existe otro conjunto independiente $J \subseteq V$ tal que $I \subset J$, es decir que $I$ est\'a incluido estrictamente en $J$. Todo conjunto independiente maximal es un \emph{conjunto dominante}.
 
\subsubsection*{Demostraci\'on}

Sea G = (V, E), $I \subseteq V$ independiente maximal.

Quiero ver que:
Para todo v\'ertice de V, o bien v pertenece a I, o bien alg\'un vecino de v pertenece a I.
%($\forall$ v $\in$ V) (v $\in$ I $\vee$ $\exists$ w $\in$ adyacentes(v) tal que w $\in$ I)\\
Como I es independiente, no existe ning\'un eje de E entre v\'ertices de I; por ser I maximal, no existe ning\'un v\'ertice w de V que no pertenezca a I, tal que w no sea adyacente a alguno de I.\\
Entonces:
\begin{itemize}
\item si E = $\emptyset$, I = V y por lo tanto cada elemento de v pertenece a I trivialmente
\item si E $\neq$ $\emptyset$, todos los v\'ertices v $in$ V que no pertenezcan a I, no lo hacen porque ya existe w $in$ I que verifica que el eje (v,w) $in$ E. Luego, cada u v\'ertice de V o bien pertenece a I, o bien es adyacente a alg\'un elemento de I. Por lo tanto I es un conjunto dominante.
\end{itemize}
\textcolor{blue}{se fijan que onda?}
\textcolor{red}{Demostracion magica negra turbia}

\subsection{Situaciones de la vida real}

Situaciones de la vida real que puedan modelarse utilizando CIDM:

\textcolor{red}{Otra vez Noriega salvando las papas del fuego..} lo del aula en el examen y eso