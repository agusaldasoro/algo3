\section{Dakkar}
\subsection{Descripci\'on de la problem\'atica}
La problem\'atica trata de una traves\'ia, la cual cuenta con \emph{n} cantidad de etapas. Para cada una de las etapas, se puede elegir recorrerla en alguno de los tres veh\'iculos disponibles: una BMX, una motocross o un buggy arenero. Cada uno de ellos permite concretar cada etapa en cantidades de tiempo diferentes. Adem\'as, la cantidad de veces que se pueden usar la motocross y el buggy arenero est\'a acotada por \emph{k}$_m$ y \emph{k}$_b$ respectivamente.

Los \emph{tiempos} que le llevan a los veh\'iculos recorrer el trayecto var\'ian por cada etapa y son datos conocidos pasados por par\'ametro.

Se pide recorrer la traves\'ia, dentro de las restricciones, de modo que se utilice la menor cantidad de tiempo posible. Si existen dos (o m\'as) maneras de atravesarla dentro del tiempo \'optimo, se pide devolver s\'olo una.

Se exige resolver la problem\'atica con una complejidad temporal de $O(n.k_m.k_b)$.\\

\textcolor{red}{Dibujitos con ejemplos :)}

\newpage
\subsection{Resoluci\'on propuesta y justificaci\'on}

Para resolver esta problem\'atica, optamos por implementar un algoritmo de \emph{Programaci\'on Din\'amica}.\\

Con el fin de encontrar el recorrido factible que emplee menos tiempo; debemos comparar, para cada etapa, cu\'al es el menor tiempo con el que puede recorrer el camino faltante eligiendo en la instancia actual uno de los tres veh\'iculos disponibles. Dado que la formulaci\'on de este problema es muy extensa, se realiz\'o una formulaci\'on recursiva de modo que para cada problema se le asigna un valor dependiendo de un subproblema menor. 

\subsubsection*{Formulaci\'on Recursiva}

Optamos por comenzar recorriendo desde la etapa n hasta la etapa 0; \emph{n} va a indicar la etapa actual, \emph{k$_m$} la cantidad de motos y \emph{k$_b$} la cantidad de buggys restantes que se pueden utilizar.

\begin{itemize}
\item[•]Cuando llegamos a la etapa \emph{n}$=0$ es porque terminamos todo el recorrido, de modo que el tiempo devuelto va a ser 0.

\item[•]Cuando \emph{k$_m$}$=0$ y \emph{k$_b$}$=0$ es porque la etapa actual (\emph{n}) y el recorrido restante (las \emph{n-1} etapas) lo vamos a tener que hacer s\'olo en bicicleta, sin importar el tiempo que conlleve ya que nos quedamos sin motos y buggys para usar.

\item[•]Cuando \emph{k$_m$}$=0$ y \emph{k$_b$}$\neq0$ es porque utilizamos la mayor cantidad de motos posibles y las \emph{n-1} etapas restantes -conjunto a la actual(\emph{n})- las vamos a tener que recorrer con Bicicleta o buggy. Por este motivo se elige la opci\'on con tiempo menor usando Bicicleta o buggy en la etapa \emph{n} y llamando recursivamente a la funci\'on para \emph{n-1} considerando esta elecci\'on.

\item[•]De modo an\'alogo, cuando \emph{k$_m$}$\neq0$ y \emph{k$_b$}$=0$ s\'olo vamos a contar con Motos y Bicicletas para la etapa actual y las \emph{n-1} etapas faltantes.

\item[•]En cambio, en caso contrario, todav\'ia tenemos disponible cantidad de los tres veh\'iculos. Por este motivo, se comparan los tres casos: empleando la Bicicleta en la etapa n, la Moto o el buggy llamando recursivamente a la funci\'on para \emph{n-1} de modo que va a devolver el menor tiempo posible considerando la elecci\'on llevada a cabo.
\end{itemize}



\begin{equation*}
func(n, k_m, k_b) = 
\begin{cases} 
       0  & \mbox{si } n = 0  \\[2ex]
       tiempoBici(n) + f(n-1, 0, 0)  & \mbox{si } k_m=0 \wedge k_b=0 \\[2ex]
      min \left(
      \begin{split}
       tiempoBici(n) & + func(n-1, 0, k_b) , \\
       tiempobuggy(n) & + func(n-1, 0, k_b-1)
\end{split} \right) & \mbox{si } k_m=0 \wedge k_b\neq0 \\[3ex]
      min \left(
      \begin{split}
       tiempoBici(n) & + func(n-1, k_m, 0) , \\
       tiempoMoto(n) & + func(n-1, k_m-1, 0)
\end{split} \right) & \mbox{si } k_m\neq0 \wedge k_b=0 \\[3ex]
           min \left(
      \begin{split}
       tiempoBici(n) & + func(n-1, k_m, k_b) , \\
       tiempoMoto(n) & + func(n-1, k_m-1, k_b) , \\
       tiempobuggy(n) & + func(n-1, k_m, k_b-1)
\end{split} \right) & \mbox{sino}
\end{cases} 
\end{equation*}

Dado que los $n$, $k_m$ y $k_b$ iniciales van a ser los dados por par\'ametro y en el planteo de nuestra ecuaci\'on en la llamada recursiva n siempre decrementa en 1 y los dem\'as o bien quedan iguales o uno de ellos decrementa en uno, estos par\'ametros van a estar acotados por:

\begin{equation*}
\begin{array}{lllll}
0 & \leq & n &\leq & n_{parametro} \\
0 & \leq & k_m & \leq & k_{m_{parametro}} \\
0 & \leq & k_b & \leq & k_{b_{parametro}}
\end{array}
\end{equation*}

\newpage
Esta formulaci\'on recursiva resuelve el problema porque.... \textcolor{red}{Poner aca algo del estilo de la demo de golosos je} \textcolor{blue}{ALGUIEN???!!!}

\subsubsection*{Diccionario a utilizar}

El diccionario que vamos a utilizar consiste en una matriz de $k_{m_{inicial}}$\texttt{x}$k_{b_{inicial}}$, en la que por cada posici\'on va a haber un contenedor de tama\~no $n_{inicial}$ (lo cual forma un cubo) de modo que dentro de cada uno de ellos se va a poder almacenar el resultado de invocar a la funci\'on con estos tres par\'ametros.

\subsubsection*{Formulaci\'on Top Down}

	Si analizamos el comportamiento de nuestra funci\'on recursiva como si fuera un algoritmo recursivo, podemos notar que la primer posici\'on del cubo que va a poder completar va a ser la de $(k_{m_{inicial}},k_{b_{inicial}},0)$. Dado que para todas las guardas el primer caso que compara es usar la bicicleta. \textcolor{red}{Explicar con un poco mas de detalle.}

	Luego, seguir\'a completando la matriz variando en orden ascendente el tercer par\'ametro y completando para todos los $k_m$ y $k_b$.

\subsubsection*{Formulaci\'on Bottom Up}


Una vez comprendido el comportamiento de la funci\'on, podemos establecer una manera de completar nuestro diccionario cubo de modo iterativo.\\

Primero completamos para $n=0$ todos los valores de $k_m$ y $k_b$. Notar que dados los \'indices que utilizamos a la hora de la implementaci\'on, cuando hablamos de $n=0$ nos referimos a la etapa 1 de la carrera (la \'ultima que vamos a recorrer), lo cual no es estrictamente el mismo resultado que el expuesto en la formulaci\'on recursiva. Sin embargo, podemos denotar esto como una diferencia que no afecta al funcionamiento del algoritmo.

Cuando $n=0$ simplemente vamos a asignar el veh\'iculo que presente menor tiempo para esta etapa, entre los disponibles bas\'andonos en los $k_m$ y $k_b$.

Luego para $n=1$, debemos considerar el costo m\'inimo entre usar cada uno de los veh\'iculos disponibles para esta etapa sumado a su correspondiente costo m\'inimo de la etapa anterior.

Continuamos con las iteraciones hasta llegar a $n=n_{parametro}$.\\

\textcolor{red}{Aca poner un dibujito que ejemplifique.}\\

Cuando contamos con la matriz Diccionario completa, nuestro resultado va a estar ubicado en la posici\'on $(k_{m_{parametro}},k_{b_{parametro}},n_{parametro})$.\\

Al ingresar los datos dentro del diccionario de esta manera, lo \'unico que hacemos es invertir el orden de llenado. Es por este motivo que estamos cumpliendo con el mismo comportamiento de la funci\'on recursiva ya mencionada. Como ya pudimos asegurar que la formulaci\'on recursiva resolv\'ia de manera correcta la problem\'atica planteada, estamos en condiciones de afirmar que completar el diccionario en el orden inverso tambi\'en resuelve el ejercicio.\\
\\


%%%%%%%%%%%%%%%%
Con el fin de poder calcular el camino una vez obtenida la cantidad de tiempo menor para realizar el trayecto, en cada posici\'on del diccionario no s\'olo guardamos el tiempo empleado para llegar hasta ah\'i sino que tambi\'en cu\'al fue su situaci\'on inmediatamente anterior.\\

De este modo, recorremos la matriz  desde la posici\'on $(k_{m_{parametro}},k_{b_{parametro}},n_{parametro})$ hacia la situaci\'on que indica la misma, y as\'i sucesivamente. De este modo obtenemos las elecciones hechas desde la \'ultima etapa hasta la primera.



\newpage
\subsection{An\'alisis de la complejidad}
\subsubsection{Complejidad Temporal}

Como par\'ametro vamos a recibir $m_k$, $m_b$, $n$ y los tiempos necesarios para atravesar con cada veh\'iculo las \emph{n} etapas. Los tiempos los vamos a tener almacenados en un vector de tuplas, donde cada componente va a ser BMX, Moto y buggy.\\

Mediante tres \texttt{for}s anidados (primero por la cantidad de etapas, despu\'es por la cantidad de motos y por \'ultimo la cantidad de buggys ), vamos a recorrer posici\'on a posici\'on nuestra matriz de vectores. Comenzando en la posici\'on $(0,0,0)$ y finalizando en  $(k_{m},k_{b},n)$.

Dentro de cada iteraci\'on, lo que vamos a hacer es lo que indica el planteo recursivo, con la salvedad de que cada casillero representa cu\'anto nos cuesta llegar a la siguiente etapa, es decir que la posici\'on (0,0,0) tendr\'a lo que cueste ir de la etapa 0 a la etapa 1, usando la bici.

Siempre escribiremos en el diccionario en la posici\'on $(k{_m}, k{_b}, n)$, omitimos la escritura para devolver el camino, pero aclaramos que son operacion elementales (asignacion y suma de enteros) y un llamado a make_pair \textcolor{blue}{link plis :)} que toma tiempo constante:

\begin{itemize}
\item Si $n=0 \wedge k_m=0 \wedge k_b=0$ escribimos el costo de la bici para ir de 0 a 1
\item Si $n=0 \wedge k_m=0 \wedge k_b\neq0$ escribimos el m\'inimo entre la bici y el buggy
\item Si $n=0 \wedge k_m\neq0 \wedge k_b=0$ escribimos el m\'inimo entre la bici y la moto
\item Si $n=0 \wedge k_m\neq0 \wedge k_b\neq0$ escribimos el m\'inimo entre la bici, la moto y el buggy
\item Si $n\neq0 \wedge k_m=0 \wedge k_b=0$ escribimos el costo de la bici + $dicc(m, b, n-1)$
\item Si $n\neq0 \wedge k_m=0 \wedge k_b\neq0$ escribimos el m\'inimo de la bici + $dicc(m, b, n-1)$ y el buggy + $dicc(m, b-1, n-1)$
\item Si $n\neq0 \wedge k_m\neq0 \wedge k_b=0$ escribimos el m\'inimo de la bici + $dicc(m, b, n-1)$ y la moto + $dicc(m-1, b, n-1)$
\item Si $n\neq0 \wedge k_m\neq0 \wedge k_b\neq0$ escribimos el m\'inimo de la bici + $dicc(m, b, n-1)$, la moto + $dicc(m-1, b, n-1)$ y el buggy + $dicc(m, b-1, n-1)$
\end{itemize}

Finalmente retornamos el valor de $dicc(k_{m_{parametro}},k_{b_{parametro}},n_{parametro})$\\

La complejidad resulta entonces: por cada etapa, por cada posible uso de moto, por cada posible uso de buggy, hacemos cosas en tiempo constante. De esto deducimos que la complejidad es de $O(k_{m},k_{b},n)$ como se ped\'ia en el enunciado.

\subsubsection{Complejidad Espacial}
Dado que la estructura utilizada es una matriz (\texttt{vector$<$vector$>$}) de $m_k$x$m_b$, donde en cada posici\'on contamos con un arreglo de $n$ posiciones y una tupla ($O(1)$); su complejidad espacial es de $\mathbf{O(m_k.m_b.n)}$.

\newpage
\subsection{C\'odigo fuente}
\subsection{Experimentaci\'on}

\subsubsection{Constrastaci\'on Emp\'irica de la complejidad}
